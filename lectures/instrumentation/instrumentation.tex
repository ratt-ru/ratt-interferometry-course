\documentclass[usenatbib,usegraphicx]{article}

\usepackage[usenames,dvipsnames]{color}
\usepackage[xindy]{glossaries}
\usepackage{graphicx}
\usepackage{multirow}
\usepackage{subfigure}
\usepackage{fullpage}

\makeglossaries

%glossary
\newacronym{alma}{ALMA}{Atacama Large Millimeter Array}
\newacronym{db}{dB}{decibels}
\newacronym{kat}{KAT-7}{Karoo Array Telescope}
\newacronym{vla}{VLA}{Very Large Array}
\newacronym{wsrt}{WSRT}{Westerbork Synthesis Radio Telescope}

\begin{document}

\title{Instrumentation for Radio Astronomy}
\author{Griffin Foster}
\date{\today}
\maketitle

\label{firstpage}

When thinking about engineering and the development of an instrument from a practical point of view the key concept to get across is that \emph{any measurement is a loss in information}.
In the case of astronomy, there is the potential to measure infinite frequency bandwidth, frequency resolution, time bandwidth, and time resolution across a $4\pi$ area of the sky.
In reality, the instruments that we can build have limits on all these parameters, so we must be selective about which information we retain based on the scientific goals and engineering limitations.

\section{Classic (dish-based) Interferometry}

We start with dish-based interferometry, which I am calling `classic' interferometry.
This is a collection of antennas which have some sort of curved optics, are mechanically pointed, and correlated together to form visibilites.
Examples of this type of array are the \gls{vla}, \gls{kat}, \gls{alma}, and \gls{wsrt}.
Using this type of interferometric array, we will follow the path of electromanetic waves from the interface with the dish optics up to the final corrrelated visibilities.

%FIG: dish-based array photos

\begin{itemize}
\item `classic' interferometry using pointable dish optics, e.g. VLA and KAT-7/MeerKAT
\item starting from the optics that focus the EM-waves, end up at the stored visibilities
\end{itemize}

\subsection{Optics and EM Interactions}

Given a dish diameter $D$, and an observing wavelength $\lambda$ we think of two different interaction domains.
When $D \gg \lambda$ light can be treated as a particle, this is the case for optical astronomy.
For example, the ratio of $D/\lambda$ for a small optical telescope with a $0.5$ meter mirror, obsvering visible light (500 nm) is $1000000$.
When $D \sim \lambda$ (with in a few orders of magnitude at least) then light needs to be treated as a wave, with complex interactions, this is the case for radio astronomy.
For example, a \gls{vla} dish is 25 meters in diameter, observing at L-band (21 cm), then $D/\lambda = 120$, in the limit where the light is wave like.
Due to the scale of radio waves (10 m - 1 cm), and the inherent scale of human 

\begin{itemize}
\item directivity: we are choosing to prefer signal from one direction over another. coherent and incoherent structure leads to directivity, but the sky is always the same temperature
\item EM scale for radio waves, bandwidth
\item EM interactions lead to complex patterns and caustics, compared to optical where the scales are much different
\item dish accuracy and efficiency
\end{itemize}

\subsection{The Primary Beam}

\begin{itemize}
\item basic primary beam of an ideal dish, airy disk from top-hat function
\item sidelobes
\item introduction of structure, practical primary beams like VLA
\item how this effects sources: the vast majority of radio sources are stable on observing timescales, but we see variation as the source moves through the beam. beams are physical, depending on where you point, how much wind is blowing, ... lead to deformations
\end{itemize}

\subsection{Antenna Feeds}

\begin{itemize}
\item EM waves induce a current, which can then be recorded/digitized
\item simple quarter-wave dipole is good for a monochromatic signal
\item differtent types of antennas are used for different characteristics, there is no ideal: log-periodic, circular, dipoles, ...
\item polarization
\item impedence matching
\end{itemize}

\subsection{Analogue Receiver Front-end}

\begin{itemize}
\item or back-end, depending on whom you ask and in what context
\item LNAs
\item cryo-stats
\item sky and system temperature
\item amplifier/filter chains
\end{itemize}

\subsection{Digitization}

\begin{itemize}
\item ADC: the last analogue component
\item bit resolution: wider bandwidth -> increase in power -> requires more bits in high RFI environment
\end{itemize}

\subsection{Spectrometers}

\begin{itemize}
\item the FFT
\item signal conditioning with a FIR to create the PFB to reduce spectral leakage (RFI mainly)
\end{itemize}

\subsection{Correlators}

\begin{itemize}
\item correlation theory
\item FX design
\item XF design
\item implementations: FPGA, GPU, CPU
\item keep the cross-correlations, chuck the autos
\item extras: delay and fringe-stopping
\item bit quantization
\item integration
\item output visibilities
\end{itemize}

\section{Aperture Arrays}

\begin{itemize}
\item a generalization: a dish interfereometer without the dish
\item PAPER, LOFAR
\item sparse and dense aperture arrays
\item generally used for low frequency
\item PAFs are essentially AAs at the center of a dish feed
\end{itemize}

advantages:
\begin{itemize}
\item access to a larger subset of the sky
\item can form multiple pointing directions
\item science advantages to short baselines
\item cheaper construction, don't need a large mechanical system to phycisally point
\end{itemize}

disadvantages:
\begin{itemize}
\item sources transit the elements, so the primary beam is more important to know
\item many more electronic components for the same collecting area
\item as the frequency scales up a dense aperture array becomes sparse
\end{itemize}

\subsection{Beamforming}

\begin{itemize}
\item forming a new primary beam out of multiple elements
\item time-domain
\item frequency-domain
\item hierarchical
\item incohrent beamforming for sky coverage
\end{itemize}

\section{VLBI}

\begin{itemize}
\item dish interferometer on very large scales -> high resolution
\item technical issues: time distribution, station position, data transport and correlation
\end{itemize}

\section{Single-Dish Astronomy}

\begin{itemize}
\item measuring the auto-correlation
\item relation to interferometry: sampling the large scale structure down to the sky DC
\item technical issues: self-noise, RFI
\end{itemize}

\section{The SKA}

\begin{itemize}
\item what is purposed to be built: LFAA, MIDAA, MID-DISH, MID-SURVEY
\end{itemize}

\label{lastpage}

\end{document}

